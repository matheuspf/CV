\documentclass[a4paper,10pt]{article}

%A Few Useful Packages
\usepackage{marvosym}
\usepackage{fontspec} 					%for loading fonts
\usepackage{xunicode,xltxtra,url,parskip} 	%other packages for formatting
\RequirePackage{color,graphicx}
\usepackage[usenames,dvipsnames]{xcolor}
\usepackage[big]{layaureo} 				%better formatting of the A4 page
% an alternative to Layaureo can be ** \usepackage{fullpage} **
\usepackage{supertabular} 				%for Grades
\usepackage{titlesec}					%custom \section

%Setup hyperref package, and colours for links
\usepackage{hyperref}
\definecolor{linkcolour}{rgb}{0,0.2,0.6}
\hypersetup{colorlinks,breaklinks,urlcolor=linkcolour, linkcolor=linkcolour}

%FONTS
\defaultfontfeatures{Mapping=tex-text}
%\setmainfont[SmallCapsFont = Fontin SmallCaps]{Fontin}
%%% modified for Karol Kozioł for ShareLaTeX use
\setmainfont[
SmallCapsFont = Fontin-SmallCaps.otf,
BoldFont = Fontin-Bold.otf,
ItalicFont = Fontin-Italic.otf
]
{Fontin.otf}
%%%

%CV Sections inspired by: 
%http://stefano.italians.nl/archives/26
\titleformat{\section}{\Large\scshape\raggedright}{}{0em}{}[\titlerule]
\titlespacing{\section}{0pt}{3pt}{3pt}
%Tweak a bit the top margin
%\addtolength{\voffset}{-1.3cm}

%Italian hyphenation for the word: ''corporations''
\hyphenation{im-pre-se}

%-------------WATERMARK TEST [**not part of a CV**]---------------
\usepackage[absolute]{textpos}

\setlength{\TPHorizModule}{30mm}
\setlength{\TPVertModule}{\TPHorizModule}
\textblockorigin{2mm}{0.65\paperheight}
\setlength{\parindent}{0pt}

%--------------------BEGIN DOCUMENT----------------------
\begin{document}

%WATERMARK TEST [**not part of a CV**]---------------
%\font\wm=''Baskerville:color=787878'' at 8pt
%\font\wmweb=''Baskerville:color=FF1493'' at 8pt
%{\wm 
%	\begin{textblock}{1}(0,0)
%		\rotatebox{-90}{\parbox{500mm}{
%			Typeset by Alessandro Plasmati with \XeTeX\  \today\ for 
%			{\wmweb \href{http://www.aleplasmati.comuv.com}{aleplasmati.comuv.com}}
%		}
%	}
%	\end{textblock}
%}

\pagestyle{empty} % non-numbered pages

\font\fb=''[cmr10]'' %for use with \LaTeX command

%--------------------TITLE-------------
\par{\centering
		{\Huge Matheus \textsc{Pedroza Ferreira}
	}\bigskip\par}

%--------------------SECTIONS-----------------------------------
%Section: Personal Data
\section{Dados Pessoais}

\begin{tabular}{rl}
    %\textsc{Place and Date of Birth:} & Someplace, Italy  | dd Month 1912 \\
    \textsc{Endereço:}   & Quadra 308 Sul, Alameda 02, Lote 02, Palmas \\
    \textsc{Telefone:}   & (35) 98426-2535 \\
    \textsc{email:}      & \href{mailto:matheuspff@gmail.com}{matheuspff@gmail.com} \\
    \textsc{Idade:}      & 22 \\
\end{tabular}\\




% ------------------------------------------------------------------ %

\section{Artigos}

\begin{tabular}{p{11cm}|r}
  Programação Linear Aplicada a Dietas que Previnam Anemia  & \textsc{Agosto 2014} \\
  e Hipovitaminose A em Crianças com Idade de 0 a 3 Anos & \\
  \textsc{Apresentação de trabalho de iniciação científica} & \\
  \footnotesize{Análise de dietas baseada em programação linear} & \\
\end{tabular}\\

\begin{tabular}{p{11cm}|r}
  Analysis of Guava Quality by Image Processing & \textsc{Dezembro 2016} \\
  \textsc{International Journal of Computer Applications} & \\
  \footnotesize{Aplicação de visão computacional para determinação automática da qualidade de goiabas} & \\
\end{tabular}\\

\begin{tabular}{p{11cm}|r}
  Improvements on Biased Random-Key Genetic Algorithms for & \textsc{Julho 2017} \\
  Non-Linearly Constrained Global Optimization & \\
  \textsc{European Journal of Operational Research} (Sob revisão) & \\
  \footnotesize{Melhoramento significativo sobre a meta-heurística BRKGA para a resolução de problemas de otimização com restrições não lineares quaisquer.} & \\
\end{tabular}\\

\begin{tabular}{p{11cm}|r}
  Parameter Optimization for JSEG Image Segmentation & \textsc{Janeiro 2018} \\
  Applied to Detection of Defective Fruits & \\
  \textsc{Computers and Electronics in Agriculture} (Sob análise dos revisores) & \\
  \footnotesize{Otimização dos parâmetros do algoritmo de segmentação JSEG por meio de simulated annealing para análise de frutos de forma automática} & \\
\end{tabular}\\

\begin{tabular}{p{11cm}|r}
  A Constrained ITGO Heuristic Applied to Engineering Optimization & \textsc{Novembro 2017 - Atual} \\
  \textsc{Expert Systems With Applications} (Sob análise dos revisores) & \\
  \footnotesize{Desenvolvimento de uma meta-heurística apresentando resultados de estado da arte em problemas de engenharia restritos.} & \\
\end{tabular}\\


% ------------------------------------------------------------------ %

\section{Linguagens de Programação e Sistemas Operacionais}
\begin{tabular}{rl}
Proficiência / Experiência profissional em: & \textsc{C}, \textsc{C++}, \textsc{Python}, \textsc{Matlab}, \textsc{Octave}, {\fb \LaTeX}, \textsc{Linux} \\
Conhecimentos intermediários em: & \textsc{Javascript}, \textsc{Typescript}, \textsc{Html}, \textsc{CSS}, \textsc{Vue}, \textsc{Windows} \\
\end{tabular}\\

% ------------------------------------------------------------------ %

\section{Iniciação Científica}

\begin{tabular}{p{11cm}|r}
  Programação Linear Aplicada a Dietas que Previnam Anemia  & \textsc{2013 - 2014}  \\
  e Hipovitaminose A em Crianças com Idade de 0 a 3 Anos & \\
  PIBIC - UFT & \\
  Orientador: Prof. Dr. Warley Gramacho da Silva & \\ 
  \footnotesize{Análise de dietas baseada em programação linear} & \\
\end{tabular}\\

\begin{tabular}{p{11cm}|r}
  Análise de qualidade de goiabas por meio de processamento de imagens  & \textsc{2014 - 2016} (renovação)  \\
  PIBIC - CNPq & \\
  Orientador: Prof. Dr. Warley Gramacho da Silva & \\ 
  \footnotesize{Análise automática de qualidade de goiabas por meio de algoritmos de segmentação e classificação} & \\
\end{tabular}\\

\begin{tabular}{p{11cm}|r}
  Desenvolvimento de uma framework genérica para  & \textsc{2016 - 2018} (renovação)  \\
  algoritmos evolutivos & \\
  PIBIC - CNPq & \\
  Orientador: Prof. Dr. Marcelo Lisboa Rocha & \\ 
  \footnotesize{Desenvolvimento de uma framework em C++ para a criação de algoritmos evolutivos.} & \\
\end{tabular}\\

% ------------------------------------------------------------------ %

\section{Premiações}

\begin{tabular}{p{11cm}|r}
  Medalha de bronze na olimpíada brasileira de matemática & \textsc{2012} \\
  das escolas públicas &  \\
\end{tabular}\\

\begin{tabular}{p{11cm}|r}
  Campeão da região norte na XX maratona de programação (ICPC) & \textsc{2015} \\
\end{tabular}\\

\begin{tabular}{p{11cm}|r}
  Campeão da região norte na XXI maratona de programação (ICPC) & \textsc{2016} \\
\end{tabular}\\

% ------------------------------------------------------------------ %

\section{Monitorias e Tutorias}

\begin{tabular}{p{11cm}|r}
  Monitor de introdução à programação e introdução à ciência da computação & \textsc{2016.2} \\
  UFT - Ciência da computação & \\
\end{tabular}\\

\begin{tabular}{p{11cm}|r}
  Tutor de introdução à programação & \textsc{2017.2} \\
  UFT - Ciência da computação & \\
\end{tabular}\\


% ------------------------------------------------------------------ %

\section{Formação}

\begin{tabular}{p{11cm}|r}
  Bacharelado em Ciência da Computação - UFT (2013 - 2018) & \textsc{Término em Julho de 2018} \\
\end{tabular}\\


% ------------------------------------------------------------------ %



\end{document}
